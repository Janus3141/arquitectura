
\documentclass[a4paper]{article}
\usepackage[utf8]{inputenc}
\usepackage{graphicx}
\usepackage{float}
\renewcommand{\baselinestretch}{1}
\usepackage[margin=3.5cm]{geometry}
\usepackage{proof}
\usepackage{amssymb}
\usepackage{listings}
\usepackage{fancyhdr}
\renewcommand{\familydefault}{\sfdefault}
\lstset{language=Haskell,
        basicstyle=\small,
        showspaces=false,
        showstringspaces=false}
\pagestyle{fancy}
\lhead{Corrutinas avanzadas}
\rhead{Bruno Sotelo}

\setlength{\parindent}{0in}

\begin{document}

\raggedright

\begin{titlepage}
\centering
\begin{figure}[H]
    \begin{center}
        \includegraphics[width=2cm,height=2cm]{UNR_logo.jpg}
    \end{center}
\end{figure}
{\scshape\large Facultad de Ciencias Exactas, Ingenier\'ia y Agrimensura\\*
                 Licenciatura en Ciencias de la Computaci\'on\par}
\vspace{5cm}
{\scshape\LARGE Arquitectura del computador \par}
{\huge\bfseries Trabajo Pr\'actico final \par}
{\huge\bfseries Corrutinas avanzadas \par}
\vspace{3cm}
{\Large Sotelo, Bruno\par}
\vfill
{\large 18  / 12 / 2017 \par}
\end{titlepage}



\section{Introducción}
El scheduler implementado en el trabajo es del tipo Multilevel Feedback 
Queue, se utilizan varias colas donde cada una posee una cantidad de
quantums diferente. Este tipo de scheduler ha demostrado funcionar muy
bien en sistemas en general ya que permite separar fácilmente a las
tareas que hacen gran uso del procesador y las que se centran en I/O. Al
contrario del Multilevel Queue Scheduler, donde a las tareas se les
asigna una prioridad y son insertadas en la cola con la prioridad 
correspondiente, en este scheduler las tareas pueden cambiar su prioridad,
dando lugar a un mayor dinamismo ante cambios de comportamiento de las
mismas, y evitando también starvations en las colas de más baja prioridad.
Para facilitar su uso se implementó una interfaz similar a la de los
Pthreads de C, por ejemplo: las funciones que serán alimentadas al scheduler
tienen como tipo \textbf{void *f(void *)}, para activarlas se utiliza una
función \textbf{create\_task}, y una tarea puede esperar a que otra termine
con \textbf{join\_task}. \\
En cuanto a la comunicación entre tareas, se eligió el esquema de shared
memory. La interfaz de I/O con este IPC es muy similar a la de un archivo,
cada tarea puede crear una o varias conexiones con un segmento de memoria
compartida y poseerá un puntero a esta en cada una de las conexiones. El
puntero puede moverse libremente en el área escrita de un segmento, pero
se encuentra la restricción de escribir secuencialmente (es decir, la 
parte escrita del área será una sola, sin espacios no escritos en el
medio). \\
Para realizar los locks también se usó una interfaz similar a los locks
de Pthreads, aunque muy simplificada. Se puede crear, bloquear, desbloquear
y eliminar un lock.

\section{Scheduler}
Como se dijo al principio, el scheduler utiliza múltiples colas de tareas.
Como además cada cola representa una prioridad de ejecución, una buena forma
de crearlas y mantenerlas es la estructura de \textbf{cola de prioridad}.
Esta se encuentra implementada en los archivos \textit{pqueue.c} y
\textit{pqueue.h}. Es una implementación mínima pero suficiente para este
scheduler, en ella se puede: insertar un nuevo nodo con la prioridad más
alta (\textit{pqueue\_new\_node}), reinsertar un nodo con una prioridad
cualquiera (\textit{pqueue\_insert}), quitar y poseer el primer elemento
de la cola (\textit{pqueue\_pop}) y mover todos los elementos a la mayor
prioridad (\textit{pqueue\_lift}, una función específica para este scheduler).
Esta implementación es bastante flexible y simple para ayudar a mantener
así también el código del scheduler. \\
Para comenzar a utilizarlo, primero se deberá reservar espacio para una
estructura Task. Esta contiene toda la información necesaria para
administrar la ejecución de la tarea: 
\begin{itemize}
    \item Un buffer sobre el cual se ejecuta \textit{setjmp} y
    \textit{longjmp} para controlar el hilo de ejecución,
    \item Una variable \textit{Task\_State} que mantiene el estado actual
    de la tarea. Se consideran en este trabajo cuatro estados: activa
    (la tarea está utilizando el procesador), lista (desea utilizar el
    procesador pero debe esperar su turno), bloqueada (se encuentra a
    la espera de ser desbloqueada por alguna condición) y zombie (la
    tarea ya terminó y no utilizará más el procesador, se mantiene en
    este estado para conocer su resultado, y saber que terminó).
    \item El argumento que recibe la tarea al momento de ser ejecutada
    (le es entregado a través de \textit{create\_task}).
    \item Un puntero a la función que ejecuta la tarea.
    \item Un puntero al resultado de la función ejecutada (será NULL
    si no termina).
    \item Un bit (\textit{queued}) que indica si la función se encuentra
    en alguna cola de espera del scheduler o no.
    \item Posición inicial del stack de memoria. Cada tarea posee un
    fragmento de stack sobre el cual puede actuar libremente. El tamaño
    de uno de estos fragmentos está parametrizado por una macro en
    \textit{mm.h}.
\end{itemize}
Una vez reservada una estructura \textit{Task}, se debe llamar a la
función \textit{start\_sched}, que recibe como argumento un puntero
a \textit{Task}, una tarea tratada especialmente porque representa
el hilo de ejecución principal del programa creado. En la función
se realizan varios ajustes y creaciones de variables necesarias
para poner el scheduler en marcha. Primero se setea el manejo de
señales del programa. La función \textit{scheduler},
definida también en \textit{scheduler.c}, actuará como handler de
las señales RT \textit{SIG\_TASK\_YIELD, SIG\_TASK\_NEW, SIG\_TASK\_END},
las cuales indicarán a este handler que la tarea activa le "devuelve"
el procesador, que se crea una nueva tarea y que la tarea activa
termina y no se le debe dar el procesador nuevamente, respectivamente.
Se utilizan señales real-time ya que poseen más beneficios que las
señales de POSIX-1. Luego se crean tres timers:
\begin{description}
    \item[sched\_yield\_timer]Que controlará el tiempo de ejecución que se
    le cede a una tarea; el scheduler pone en marcha este timer antes
    de poner la tarea en ejecución, y al llegar el timer a 0 mandará
    una señal \textit{SIG\_TASK\_YIELD} que lo pondrá de nuevo en marcha
    para continuar otra tarea.
    \item[sched\_lift\_timer] Que indica cada cu\'anto tiempo se llevan
    todas las tareas a la cola de mayor prioridad, lo cual evita el
    starving en las colas de menor prioridad. Est\'a parametrizado por
    \textit{SCHED\_LIFT\_SECS} y \textit{SCHED\_LIFT\_NSECS}. Este timer
    no manda señales sino que es revisado en cada ejecuci\'on del
    scheduler.
    \item[sched\_idletask\_timer] Especifica el tiempo que toma la tarea
    'idle' cuando es ejecutada. Es una tarea ociosa, fue definida para
    los casos en que no quedan tareas por ejecutar en cola. Esta tarea
    se ejecutar\'a por \textit{SCHED\_IDLETASK\_NSECS} segundos.
\end{description}
Lo siguiente en \textit{start\_sched} es setear los valores de la
estructura \textit{maintask}. El miembro \textit{mem\_position} es
iniciado con la función \textit{take\_stack}, esta es en realidad una
macro definida que abstrae a \textit{memory\_manager},
la cual controla los segmentos de stack que posee cada tarea. Puede
entregar a una nueva tarea un puntero al próximo segmento libre (que
ninguna otra está usando) o una tarea puede indicarle que no utilizará
más su segmento con la macro \textit{release\_stack}, y el manager lo
marcará como libre para cederlo a otra tarea. El tamaño de uno de
estos segmentos está parametrizado con la macro \textit{MEM\_TASK\_SIZE}\\
Lo que sigue es crear la cola de tareas del scheduler. También está
parametrizada la cantidad de colas (prioridades) que se utilizarán
con la macro \textit{QUEUE\_NUMBER}. Observar que puede crearse un
scheduler Round Robin si se define \textit{QUEUE\_NUMBER = 1}. Finalmente
se pone un checkpoint en el lugar con la macro \textit{YIELD},
básicamente una llamada a \textit{setjmp}, y se llama al scheduler.\\
El funcionamiento del scheduler en general es como sigue:
\begin{enumerate}
    \item El primer paso es desencolar la pr\'oxima tarea a ejecutar.
    Si primero se encola la tarea que se estaba ejecutando, es probable
    que esa tarea tenga una alta prioridad, y vuelva a ser elegida por
    el scheduler, lo que llevar\'ia a un starving de las dem\'as tareas.
    Esto hace necesaria la existencia de la tarea 'idle', ya que puede
    suceder que no haya otra tarea en cola, y no habr\'ia nada que
    ejecutar luego. Cuando se desencola una tarea, se debe asegurar que
    no est\'e bloqueada. Luego se explicar\'a el mecanismo de bloqueo.
    \item Si la señal que llega no es \textit{SIG\_TASK\_END} y
    \textit{qelem\_current} no es \textit{NULL}, la tarea que estaba
    en ejecución se guardará nuevamente en la cola de tareas listas
    y se hará un checkpoint (\textit{YIELD}) al cual volver cuando
    el scheduler elija otra vez esta tarea, y que implica retornar
    del handler para continuar la ejecución que se estaba dando.
    Observar que la segunda condición será falsa sólo en la primera
    llamada a \textit{scheduler}, y al finalizar una ejecuci\'on de
    la tarea 'idle'. En cuanto a la primera, su razón es
    que esa señal sólo será enviada por una tarea al terminar a
    través de la función \textit{stop\_task}, con la macro
    \textit{FINALIZE}, por lo cual no se le debe entregar otra
    vez el procesador, y no se debe encolar.
    \item Antes de encolar la tarea que estaba en ejecución se decide
    en qué nivel de prioridad se lo hace. Si la tarea utilizó todo
    el tiempo dado normalmente, se bajará en 1 su prioridad, ya que
    es probable que sea una tarea que usará más tiempo de CPU.
    En cambio, si la tarea utilizó \textit{task\_yield}, que devuelve
    el control al scheduler antes de finalizar su segmento de tiempo,
    tiene que ser porque se encuentra esperando I/O, un lock, o que
    alguna otra condición se cumpla para seguir su ejecución,
    entonces se deja a la tarea en su prioridad actual.
    \item Si la señal que llegó es \textit{SIG\_TASK\_NEW}, se deberá
    introducir, al final de la cola de espera de mayor priorida, la
    nueva tarea. Su puntero (\textit{Task *}) se pasa como argumento
    en \textit{data -> si\_value.sival\_ptr}.
    \item Cada cierto tiempo es prudente subir la prioridad de todas
    las tareas en la cola de espera a la más alta. Esto permite
    evitar starvations y mejora la dinámica del scheduler, haciendo
    que se adapte mejor a los comportamientos de las tareas. La
    acción se encuentra implementada como parte de la estructura
    \textit{pqueue} con la función \textit{queue\_lift}.
    \item Si la tarea desencolada es un puntero \textit{NULL}, es porque
    la cola estaba vac\'ia. En este caso se activar\'a la tarea 'idle'.
    \item Finalmente se configura el estado de la tarea elegida en
    \textit{ACTIVE}, se calcula la cantidad de quantums que podrá
    usar para correr, se activa el timer con ese valor y se activa
    la tarea. Los tiempos de ejecución están parametrizados con
    las macros \textit{TICK}, que es la duración en milisegundos
    de un quantum, y con \textit{QUANTUM(x)} se puede especificar
    la cantidad de quantums que obtiene una tarea de nivel $x$.
    En este caso la función de quantums es, como es recomendado,
    exponencial en función del nivel de tarea.
\end{enumerate}

\begin{figure}[H]
    \begin{center}
        \includegraphics[width=15cm,height=10cm]{state_flow.png}
        Flujo de estado de las tareas
    \end{center}
\end{figure}

Sobre el bloqueo de tareas: Si una tarea necesita bloquear a otra
debe utilizar la función \textit{block\_task}, la cual, dado un
puntero a la tarea a bloquear, simplemente cambia el estado de esa
tarea a \textit{BLOCKED}, siempre que el estado anterior no sea
\textit{ZOMBIE}. La tarea eventualmente ser\'a desencolada para ponerla
en ejecuci\'on. Cuando esto sucede, si la tarea tiene estado
\textit{BLOCKED}, procederá a eliminar el qelem correspondiente
(no la tarea) y a desencolar otra tarea que no esté bloqueada.\\
Para desbloquear una tarea se utiliza \textit{unblock\_task}.
La función le devuelve el estado
\textit{READY} y diferencia entre dos casos: la tarea está en la
cola de espera o no; si lo está, no necesita hacer nada más, el
scheduler le entregará el procesador como si nada hubiera
sucedido; si no lo está, es necesario crear un nuevo elemento
en la cola que contenga a esta tarea, el scheduler lo hará al
enviársela como si fuera una recién creada, y se la coloca con
prioridad 0 en la cola.


\section{Locks}
Los locks se encuentran implementados en \textit{locks.c}, en forma
de spinlocks (por lo cual se los nombra \textit{sp}). Los lock tienen
por tipo \textit{task\_sp\_t}, un sin\'onimo para \textit{long long},
y pueden tener dos valores, \textit{TASK\_SP\_LOCK} o 
\textit{TASK\_SP\_UNLOCK}. Para crear un lock simplemente se declara una
variable \textit{task\_sp\_t} con el valor \textit{TASK\_SP\_INIT}. Una
tarea pide un lock con \textit{task\_sp\_lock} pasando el puntero a la
variable declarada, y luego lo suelta llamando a
\textit{task\_sp\_unlock}.\\
El algoritmo para adquirir un lock se basa en la instrucci\'on de x86\_64
\textit{xchg}, la cual intercambia dos variables at\'omicamente. En la
funci\'on para adquirir el lock, primero se declara una variable con
valor \textit{TASK\_SP\_LOCK}, para intercambiarla luego con el lock
argumento, mediante \textit{xchg}. Entonces se prueba el valor de la
variable local, si sigue siendo \textit{TASK\_SP\_LOCK} es porque el
lock est\'a tomado, entonces se suelta el procesador para volver a
probar suerte en la pr\'oxima ejecuci\'on; si en cambio el valor ahora
es \textit{TASK\_SP\_UNLOCK}, es porque se adquiri\'o el lock, entonces
se puede empezar a trabajar en la secci\'on cr\'itica. Para soltar un
lock, s\'olo se le debe devolver el valor \textit{TASK\_SP\_UNLOCK}. \\
Este tipo de lock sufre algunas desventajas:
\begin{itemize}
    \item Pueden darse casos de starvation cuando una tarea no llega
    a tomar un lock. Esto se debe a que no existe un orden al pedirlo,
    las tareas se encuentran en condici\'on de carrera.
    \item Si una tarea mantiene el lock por un tiempo prolongado, se
    genera una p\'erdida de rendimiento debido a la gran cantidad de
    cambios de contextos innecesarios generados por la espera activa
    en la toma del lock.
\end{itemize}
Ambos problemas pueden ser solucionados agregando colas a la
implementaci\'on de estos locks. Si una tarea no consigue el lock al
intercambiar variables, no seguir\'a intentando conseguirlo activamente,
sino que pasar\'a a bloquearse y esperar en una cola hasta que el actual
dueño del lock lo suelte, y llame al primero en la cola. Esto se
encuentra implementado en los archivos \textit{locks2.c} y
\textit{locks2.h}. Se debe mencionar que tampoco son infalibles ya que
generan tambi\'en una ca\'ida de rendimiento cuando los tiempos de
retenci\'on de locks son muy cortos.

\section{Comunicación}
La comunicación entre procesos se realiza mediante shared memory, y
está implementada en \textit{shmem.c}. Para comenzar a utilizar un área
o región de memoria compartida, la tarea debe iniciar una conexión
usando \textit{shmem\_new} y dando el nombre del área (un caracter,
lo cual limita el número de áreas a 127). Esta función es una macro
que abstrae al manejador de memoria compartida
\textit{shmem\_region\_manager}, el cual puede crear o remover
regiones. \textit{shmem\_new} devuelve una estructura
\textit{shmem\_t}, que ser\'a utilizada como argumento para las
funciones de interfaz de shared memory. \\
El tamaño de una región de memoria compartida está parametrizado
por la macro \textit{SHMEM\_REGION\_SIZE}. Aquí se utilizan 255
bytes por región de memoria. \\
La estructura \textit{shmem\_t} contiene un puntero (*) a la región
de memoria compartida y un entero que representa el puntero de la
tarea sobre esa región. Cada región puede ser accedida por
cualquier cantidad de tareas, y cada tarea
debería iniciar su propia conexión a la región, ya que así tendrá
su puntero sobre ella. Este puntero funciona de manera similar al
que utilizan los archivos: si se leen 5 bytes de la región, el
puntero sumará 5 para que la tarea continue leyendo desde el punto
en que paró. Si se escriben 5 bytes, se los escribe a partir de la
posición actual del puntero, y este se deja en el próximo byte a
escribir. También se puede cambiar la posición del puntero con la
función \textit{shmem\_seek}; esta toma como argumentos un 
\textit{shmem\_t *}, un \textit{unsigned char} que es la posición
de puntero deseada, y un \textit{char} que puede ser:
\begin{itemize}
    \item \textit{SEEK\_ABS}, entonces la posición de puntero
    indicada es absoluta en la región (se indexa desde 0).
    \item \textit{SEEK\_OFFSET}, con el cual la posición indicada
    es relativa a la posición actual del puntero (se suma el
    argumento a la posición).
\end{itemize}
La nueva posición del puntero a la región debe estar dentro del
límite escrito (para facilitar el control, la escritura se
"serializa"), si lo está, se cambia su valor y se devuelve
\textit{SHMEM\_OK}, de lo contrario es devuelto
\textit{SHMEM\_INVALID}. \\
Las otras funciones que provee la interfaz de shared memory son:
\begin{itemize}
    \item \textit{shmem\_read} que dado el puntero a la estructura
    \textit{shmem\_t}, un puntero a un buffer $b$ y un \textit{
    unsigned char} $n$, lee de la región $n$ bytes a partir de la
    posición actual del puntero, y los guarda en $b$. Si no
    existen $n$ bytes para leer, se lee todo lo posible, y se
    devuelve la cantidad de bytes le\'idos.
    \item \textit{shmem\_write}, como read, toma el puntero a la
    región de memoria, un puntero a un buffer $b$ y un \textit{
    unsigned char} $n$, y escribe $n$ bytes de $b$ en la memoria
    compartida. Si no hay espacio para escribir $n$ bytes,
    escribe todo lo posible, y devuelve el número de bytes
    escritos.
    \item \textit{shmem\_poll} toma el puntero a \textit{
    shmem\_t} y devuelve \textit{SHMEM\_OK} si hay algo escrito
    más allá del puntero actual de la tarea, o \textit{SHMEM\_
    INVALID} si no lo hay.
    \item \textit{shmem\_block\_read} es una combinación entre
    \textit{shmem\_read} y \textit{shmem\_poll}; si no hay nada
    para leer en la región de memoria, se devuelve el control
    al CPU, para volver a consultar cuando se adquiera
    nuevamente. En cuanto haya algo para leer, se utiliza la
    función de lectura y se retorna.
    \item Observación: En todas las funciones mencionadas se
    chequea que la región aún esté activa antes de realizar
    cualquier cosa con la función \textit{shmem\_check}, si
    no lo está, se devuelve en todos los casos la bandera
    \textit{SHMEM\_CLOSED}.
\end{itemize}


\section*{Bibliograf\'ia consultada}
\begin{itemize}
    \item Andrew S. Tanenbaum. \textit{Modern Operating Systems, third edition}. 2009.
    \item Gunnar Wolf, Esteban Ruiz, Federico Bergero, Erwin Meza.
    \textit{Fundamentos de Sistemas Operativos, primera edici\'on}. 2015.
    \item Remzi H. Arpaci-Dusseau, Andrea C. Arpaci-Dusseau. \textit{Operating Systems:
    Three Easy Pieces}. 2015. 
\end{itemize}


\end{document}
